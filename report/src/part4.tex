\setcounter{secnumdepth}{1}

\chapter{Lab 4}

\begin{Task}{Question 4.1. harmonic and intermodulation distortion}
    Consider a static nonlinear system whose response is given by:\\
    \begin{equation}
        y(t) = u(t) - \frac{1}{2}u^3(t) - \frac{1}{4} u^4(t)
    \end{equation}
    At which frequencies will energy appear at the output when the input is excited by the sum of 2 sinewaves, one at frequency 4Hz and one at frequency 11Hz?    
\end{Task}

using matlab, each excited frequency has been plotted on figure \ref{fig:distortionEx}. The different amplitudes are not correct, they simply give the order of magnitude of each frequency component. A more exact simulation with the fft of the output signal is shown in figure \ref{fig:distortionExExact}.

\begin{figure}[H]
    \centering
    \includegraphics[width=0.8\textwidth]{part4/distortionEx.png}
    \caption{Excited frequencies - Approximation}
    \label{fig:distortionEx}
\end{figure}

\begin{figure}[H]
    \centering
    \includegraphics[width=0.8\textwidth]{part4/distortionExExact.png}
    \caption{Excited frequencies - Exact}
    \label{fig:distortionExExact}
\end{figure}

\begin{Task}{Question 4.2.}
    Design an excitation sinewave with a frequency of $100$ Hz and an amplitude of $1$ V. Plot the input signal and the output signal of the DUT in the frequency domain. What do you observe? Give a list of all possible solutions to get rid of this behaviour. Design experiments to show that the proposed solutions do indeed work.
\end{Task}

\begin{figure}[H]
    \centering
    \includegraphics[width=0.8\textwidth]{part4/leakage.png}
    \caption{Initial sine wave DFT}
    \label{fig:leakage}
\end{figure}

The DFT of the sinusoidal input signal is clearly not a Dirac pulse as the chosen frequency is not at an integer bin index. \\

The size of a bin being the inverse of the signal duration $N/f_s$, we want
\begin{equation*}
    f_{\text{signal}} = k \cdot \frac{f_s}{N} \quad \text{with} \quad k \in \mathbb{N}
\end{equation*}

This is visible as there is leakage in the DFT plot. There are 3 ways of solving this:
\begin{itemize}
    \item Changing the sampling frequency $\rightarrow f_s = 9990.25 \text{Hz}$
    \item Changing the number of samples $\rightarrow N = 4100 $
    \item Changing the sine frequency $\rightarrow f_{\text{signal}} = 100.1 \text{Hz}$
\end{itemize}

\begin{figure}[H]
    \centering
    \includegraphics[width=0.8\textwidth]{part4/corrected.png}
    \caption{Corrected sine wave DFT}
    \label{fig:corrected}
\end{figure}


\begin{Task}{Question 4.3.}
    Change the excitation frequency to a frequency in the neighborhood of 100 Hz in order to solve the problem in the previous step. Use 10 different amplitudes that are spaced logarithmicaly from 100 mV to 1.1 V (logspace command in MATLAB]. Plot the DFT of the output for each amplitude. What are the frequencies at which you expect distortion to appear? Are they all present in reality? Explain.
\end{Task}

\begin{figure}[H]
    \centering
    \includegraphics[width=1.1\textwidth]{part4/Task_4_3_Harmonics.png}
    \caption{Non-linear system response : Harmonics}
    \label{fig:Harmonics}
\end{figure}

\begin{figure}[H]
    \centering
    \includegraphics[width=1\textwidth]{part4/Task_4_3_Harmonics_freq_2.png}
    \caption{Non-linear system response : Harmonics frequency details}
    \label{fig:Harmonics frequencies value}
\end{figure}

By increasing the amplitude, we can observe the peaks of the harmonics rising (at frequencies which are multiple of the excited frequency which is as expected).
These peaks rise as the behaviour of the output signal deviates from a linear evolution
to a "non-linear" behaviour.  It can be easily seen by computing the output/input power graph done at the point 4.4.  It should be mentionned that even at low power,
 once the output signal deviates from a linear behaviour, the second and the third harmonics are the most visible as they have the highest contributions 
 (respectively one half and one quarter of the amplitude of the original signal frequency).

\begin{Task}{Question 4.4.}
    Plot the measurements from the previous question on an input power – output power plot. Do you see compression or expansion? From this plot, estimate the 1 dB compression or expansion point. Generate a single sine wave with the corresponding input amplitude to see how accurate your estimate is.
\end{Task}

\begin{figure}[H]
    \centering
    \includegraphics[width=1\textwidth]{part4/Task_4_4_Power_expansion.png}
    \caption{Non-linear system response : Output power non linearity}
    \label{fig:Output_power}
\end{figure}

The plot of the output/input power can be seen in the graph above.  To compute the amplitude at which the output signal shows a expansion of 1dB, the formula shown in
the picture \ref{fig:1dB_Amplitude} should be used.  The result of the formula can be seen in the matlab file "ProcessData".
In this case, the amplitude resulting from the formula is 1.0999.

\begin{figure}[H]
    \centering
    \includegraphics[width=0.8\textwidth]{part4/Task_4_4_1dB_Amplitude.png}
    \caption{Non-linear system response : Amplitude for 1dB expansion}
    \label{fig:1dB_Amplitude}
\end{figure}

ATTENTION : it should be mentionned that the point 4.3 and 4.4 are incorrect as the signal frequency has not been set according the result of the point 4.2 where a 
signal frequency of 100.1 Hz should have been computed.  Instead, due to incorrect encoding, a signal with a frequency of 104 Hz has been computed which is incorrect
as it does not fit the measurement window and therefore lead us to power leakage.



